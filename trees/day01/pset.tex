
\documentclass{article}
\usepackage[utf8]{inputenc}

\title{\large{\textsc{In-Class 13: AVL Trees}}}
\date{}

\usepackage{natbib}
\usepackage{graphicx}
\usepackage{amsmath}
\usepackage{amsfonts}
\usepackage{mathtools}
\usepackage[a4paper, portrait, margin=0.8in]{geometry}

\usepackage{listings}


\newcommand\perm[2][n]{\prescript{#1\mkern-2.5mu}{}P_{#2}}
\newcommand\comb[2][n]{\prescript{#1\mkern-0.5mu}{}C_{#2}}
\newcommand*{\field}[1]{\mathbb{#1}}

\DeclarePairedDelimiter\ceil{\lceil}{\rceil}
\DeclarePairedDelimiter\floor{\lfloor}{\rfloor}

\newcommand{\Mod}[1]{\ (\text{mod}\ #1)}

\begin{document}

\maketitle

\subsection*{Core}


\begin{enumerate}

% %%%%% PROBLEM 1 %%%%%
% \item Given this balanced AVL with nodes added in order: $[2, 1, 5, 4, 6]$, perform the necessary rotations to get the tree balanced again after the call to \texttt{insert(3)}.


% %%%%% PROBLEM 2 %%%%%
% \item Given this balanced AVL with nodes added in order: $[20,10,30,5,25,40,35,45]$, perform the necessary rotations to get the tree balanced again after the call to \texttt{insert(34)}.\\ 
% \textit{Hint: this is not the same case as above}.


%%%% POSSIBLE PROBLEM 1 %%%%%

\item For the following two cases, draw the AVL trees on the board and list the necessary rotations (eg, \texttt{rotateRight(15), rotateLeft(25), ...}) to get the tree balanced again after the call to \texttt{insert()}.

\begin{enumerate}
    \item Given this balanced AVL with nodes added in order: $[2, 1, 5, 4, 6]$, call \texttt{insert(3)}.
    
    \item Given this balanced AVL with nodes added in order: $[20,10,30,5,25,40,35,45]$, call \texttt{insert(34)}.
\end{enumerate}


%%%%% PROBLEM 3 %%%%%
\item Given the root of a BST, determine if it is a valid AVL tree.


%%%%% PROBLEM 4 %%%%%
\item Given the root of an AVL tree, return the median value in the tree.  For example, for the tree created by adding the following nodes in order: $[2, 7, 4, 9, 1, 5, 8, 3, 6]$, the median should be 5. Can you improve your solution if you're allowed to augment the data structure?

\end{enumerate}

\subsection*{Extension}


\begin{enumerate}


\setcounter{enumi}{3}

%%%%% PROBLEM 5 %%%%%
\item Given the root node of an AVL tree and a target sum integer, return true if the tree has a root to leaf path where adding up the values of every node in the path gives the target sum.


%%%%% PROBLEM 6 %%%%%
\item Convert an AVL tree to a sorted doubly-linked list by editing the pointers of the tree.  Do not create any new nodes.


\end{enumerate}

\clearpage

%%%%% SOLUTION 1 %%%%%



\begin{lstlisting}

1. 
a) Rotate right on 5 then rotate left on 4
b) Rotate right 40, rotate left 30

2.
checkValid(root):
    if root == null:
        return true
    if abs(rightChild.height - leftChild.height) >= 2:
        return false
    return checkValid(leftChild) && checkValid(rightChild)
    
3.
O(n) version

getMedian(root):
    count = iotCount(root)
    if count mod 2 == 0:
        return iotToKey(root, count/2)
    else
        return (iotToKey(root, count/2) + iotToKey(root, (count/2)+1))/2


Better version (if augmented) is to store the size 
of the left and right subtrees of each node.

4.
sumPath(root, sum):
    if node == null:
        return sum==0
    else
    
        currSum = sum - root.value
        
        if curr=0 && rightChild==null && leftChild==null:
            return sumPath(rightChild, currSum)
        if rightChild
            return sumPath(leftChild, currSum)
        else:
            return false
        
5.
    1. If left subtree exists, process the left subtree
    …..1.a) Recursively convert the left subtree to DLL.
    …..1.b) Then find inorder predecessor of root in left subtree 
    (inorder predecessor is rightmost node in left subtree).
    …..1.c) Make inorder predecessor as previous of root and root as next 
    of inorder predecessor.
    2. If right subtree exists, process the right subtree (Below 3 steps 
    are similar to left subtree).
    …..2.a) Recursively convert the right subtree to DLL.
    …..2.b) Then find inorder successor of root in right subtree 
    (inorder successor is leftmost node in right subtree).
    …..2.c) Make inorder successor as next of root and root as previous 
    of inorder successor.
    3. Find the leftmost node and return it (the leftmost node is 
    always head of converted DLL).


\end{lstlisting}

\end{document}

